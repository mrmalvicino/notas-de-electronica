\documentclass[a5paper,12pt,twoside]{book}

\input{book-packages}
\input{book-commands}

% ENTORNOS DE NUMERACIÓN
\newtheorem{defn}{{Definición}}[chapter]
\newtheorem{prop}{{Propiedad}}[chapter]
\newtheorem{example}{{Ejemplo}}[chapter]

% ENTORNOS DE FORMATO
\newenvironment{formatI}{\vspace{1ex}\par\small\sffamily} % Consignas de ejemplos

\begin{document}

\pagestyle{fancy}
\fancyhf{}
\chead{\scriptsize \nouppercase\rightmark}
\cfoot{\scriptsize \thepage}
\pagenumbering{gobble}
\renewcommand{\headrulewidth}{0pt}

\frontmatter
% \includepdf{cover}

\begin{center}
    \begin{Huge}
        \textbf{Notas de electrónica}
    \end{Huge}

    \vspace{1cm}
    \textbf{Primera edición (v20230226)}
    \vspace{2cm}

    \begin{Large}
        Malvicino, Maximiliano R.
    \end{Large}
\end{center}

\clearpage
\noindent
\textbf{Prefacio}

Resumen de contenidos de la materia electricidad y magnetismo para ingeniería de sonido de la Universidad Nacional de Tres de Febrero.

La versión digital más reciente de este texto puede ser descargada gratuitamente de \url{https://github.com/mrmalvicino/notas-de-electronica}.

\renewcommand{\spanishappendixname}{Anexo}
\tableofcontents

\mainmatter
\pagenumbering{arabic}


\chapter{Electricidad}


\section{Fuerza eléctrica}

Según el modelo atómico, la materia está formada por moléculas que están compuestas en última instancia por átomos. Pero a su vez, los átomos están formados por distintas configuraciones de \emph{partículas subatómicas}, de ahí que haya distintos tipos de elementos. Cada elemento se clasifica en la tabla periódica según su número atómico, que es la cantidad de protones que tiene un átomo de dicho elemento. Un átomo tiene la misma cantidad de protones que de neutrones, pero no necesariamente la misma cantidad de electrones. Podemos identificar así el peso de un átomo de cierto elemento, conociendo la cantidad de partículas subatómicas, según el siguiente cuadro, donde además se observa la carga eléctrica de cada una.

\begin{table}[h!]
    \begin{center}
        \begin{tabular}{|c|c|c|}
            \hline
            Partícula & Masa [\si{\kilo\gram}] & Carga [\si{\coulomb}]
            \\ \hline \hline
            Electrón & $9.1094 \times 10^{-31}$ & $-1.6022 \times 10^{-19}$
            \\ \hline
            Protón & $1.6726 \times 10^{-27}$ & $+1.6022 \times 10^{-19}$
            \\ \hline
            Neutrón & $1.6749 \times 10^{-27}$ & $0$
            \\ \hline
        \end{tabular}
        \caption{Masa y carga de las partículas atómicas.}
    \end{center}
\end{table}

La cantidad de protones (y neutrones) de un átomo permanece inalterada. Una transformación física que modificase la \emph{estructura atómica} implicaría un cambio en el núcleo del átomo haciendo que el elemento no sea el mismo. Estas transformaciones se conocen como fusión y fisión nuclear.

En cambio, la cantidad de electrones de un átomo sí puede variar sin alterar la naturaleza del elemento. Si un átomo tiene la misma cantidad de protones que de electrones se lo llama neutro. De lo contrario, se lo llama ion, o se dice que está cargado eléctricamente.

\concept{Aislantes (o dieléctricos) y conductores:}

Un dieléctrico es un material que no es conductor de la electricidad.
Para el vacío, se definen la constante de Coulomb $(\cstcoulomb)$ y la permitividad del vacío $(\epsilon_0)$.
\begin{gather*}
    \cstcoulomb=8.99 \times 10^9\,\si{\newton\metre^2\per\coulomb^2}
    \\
    \epsilon_0 = \dfrac{1}{4 \pi \cstcoulomb} = 8.85 \times 10^{-12} \,\si{\coulomb^2\per\newton\metre^2}
\end{gather*}

Según las propiedades de cada dieléctrico se puede definir una constante dieléctrica $k$ y una resistencia dieléctrica $\epsilon$ para cada material en función del paso de corriente que permitan.

En un conductor, en cambio, los electrones de los átomos que lo componen pueden moverse por los orbitales externos de cada átomo, permitiendo el paso de cargas o corriente.

Tanto los aislantes como los conductores pueden tener carga.
En caso de un aislante tener carga, esta se puede localizar en algunos de los átomos que lo componen, mientras que otros de sus átomos pueden tener carga menor o neutra.

\concept{Polarización:}

Los electrones de un objeto conductor se reacomodan internamente ante la presencia de una carga que se encuentre en su cercanía sin estar en contacto.
Esto genera que, mientras el objeto externo cargado esté presente, una mitad del conductor originalmente neutro queda con carga negativa y la otra con carga positiva, ambas equivalentes a la carga externa presente.
Si el objeto externo con carga se quita, los electrones del conductor se reacomodan para devolverle su carga neutra, ya que las cargas polarizadas son del signo contrario y se atraen entre sí.

\begin{itemize}
    \item \concept{Carga por conducción}
    
    \begin{itemize}
        \item El vidrio pierde electrones al ser frotado con seda.
        \item La goma gana electrones al ser frotada con la piel.
        \item Se dice de el vidrio queda cargado positivamente por tener más protones que electrones.
        \item Se dice que la goma queda cargada negativamente por tener más electrones que protones.
    \end{itemize}

    \item \concept{Carga por inducción}
    
    \begin{itemize}
        \item Se acerca un objeto cargado a uno neutro que está conectado a tierra.
        \item Las cargas son empujadas y se ``fugan'' por la conexión a tierra.
        \item Se corta la conexión a tierra y se aleja el objeto cargado.
        \item El objeto, antes neutro, perdió las cargas empujadas quedando cargado.
    \end{itemize}

    \begin{center}
        \def\svgwidth{\linewidth}
        \input{./images/elec-conduc-induc.pdf_tex}
    \end{center}
\end{itemize}

Los objetos cargados se repelen o atraen entre sí por acción de una fuerza.
La fuerza eléctrica entre dos cargas puntuales está dada por la \emph{Ley de Coulomb} a continuación.

\begin{mdframed}[style=MyFrame1]
    \begin{defn}
    \end{defn}
    \cusTi{Ley de Coulomb}
    \begin{equation*}
        \Vec{F}_e = \cstcoulomb \, \frac{ q_0 \, q_1}{r^2} \, \versor{r}
    \end{equation*}
\end{mdframed}


\section{Campo eléctrico}

\begin{mdframed}[style=MyFrame1]
    \begin{defn}
    \end{defn}
    \cusTi{Campo eléctrico de carga puntual}
    \cusTe{Campo vectorial generado por un cuerpo con carga que indica la fuerza eléctrica que sufriría cierta carga $q_0$ en un punto del espacio.}
    \begin{equation*}
        \Vec{E} = \frac{\Vec{F}_e}{q_0}
    \end{equation*}
\end{mdframed}

Según el principio de superposición, el campo eléctrico generado por $\nth$ cargas puntuales, va a estar dado por la suma de la contribución que cada carga genere:
\begin{equation}
    \Vec{E}= \cstcoulomb \sum_{\ith=1}^\nth \frac{q_\ith}{r_\ith^2} \, \versor{r}_\ith
\end{equation}

Ahora bien, el campo eléctrico generado por una carga contínua se define a partir del diferencial $\dif q$ que determina la distribución geométrica de la carga.

Según esté la carga dispuesta en una, dos o tres dimensiones se tendrá una densidad de carga lineal, superficial o volumétrica respectivamente. Por lo tanto $\dif q$ será, en cada caso:
\begin{align*}
    \dif q &= \lambda \, \dif x
    \\[1ex]
    \dif q &= \sigma \, \dif A
    \\[1ex]
    \dif q &= \rho \, \dif V
\end{align*}

Y si la carga está distribuída uniformemente, la densidad de carga es igual a la carga total $Q$ sobre el largo, área o volumen del material:
\begin{align*}
    \lambda &= \frac{Q}{L}
    \\[1ex]
    \sigma &= \frac{Q}{A}
    \\[1ex]
    \rho &= \frac{Q}{V}
\end{align*}

Se consideran particiones que van a tener cierta carga $\Delta q_\ith$ y se las trata como puntuales. Sumando el aporte de cada $\Delta q$ se obtiene una aproximación del campo eléctrico debido a la carga total:
\begin{equation*}
    \Vec{E} \approx \cstcoulomb \sum_{\ith=1}^\nth \frac{\Delta q_\ith}{r_\ith^2} \, \versor{r}_\ith
\end{equation*}

Luego, tomar el límite cuando $\nth\to\infty$ en la ecuación anterior equivale a hacer infinito el número de particiones. De esta forma, el aporte de cada trozo de carga será $\Delta q_\ith = \dif q$ pudiendo definir:

\begin{mdframed}[style=MyFrame1]
    \begin{defn}
    \end{defn}
    \cusTi{Campo eléctrico de carga contínua}
    \begin{equation*}
        \Vec{E} = \cstcoulomb \int \frac{\dif q}{r^2} \, \versor{r}
    \end{equation*}
\end{mdframed}

Si bien la definición es siempre válida, la integral solo se puede computar en aquellos puntos que sea posible identificar:
\begin{itemize}
    \item Una simetría que permita escribir $\versor{r}$ como uno de los versores canónicos.
    \item Una distancia $r$ tal que $\dif q$ varíe en una sola dimensión.
\end{itemize}

\begin{mdframed}[style=MyFrame2]
    \begin{example}
    \end{example}
    \cusTi{Campo eléctrico de una barra}
    \begin{formatI}
        Calcular el campo eléctrico en el punto $\vec{x}_0$ generado por una barra aislante con densidad de carga lineal uniforme. Suponer que la barra tiene grosor nulo.
    \end{formatI}

    \begin{center}
        \def\svgwidth{\linewidth}
        \input{./images/elec-barra.pdf_tex}
    \end{center}

    Observar que $r$ varía linealmente a medida que se recorre la barra.
    \begin{equation*}
        \vec{E} = \cstcoulomb \int_{d}^{d+l} \frac{\lambda \, \dif x}{x^2} \bb{- \iVer}
    \end{equation*}
\end{mdframed}

\begin{mdframed}[style=MyFrame2]
    \begin{example}
    \end{example}
    \cusTi{Campo eléctrico de un anillo}
    \begin{formatI}
        Calcular el campo eléctrico en el punto $\vec{x}_0$ generado por un anillo aislante con densidad de carga lineal uniforme. Suponer que el anillo tiene grosor nulo.
    \end{formatI}

    \begin{center}
        \def\svgwidth{\linewidth}
        \input{./images/elec-anillo.pdf_tex}
    \end{center}

    Dado que $r$ no varía,
    \begin{equation*}
        \vec{E} = \cstcoulomb \, \frac{\versor{r}}{r^2} \int \dif q
    \end{equation*}
    
    Donde:
    \begin{equation*}
        \left\{
        \begin{aligned}
            & \int \dif q = Q
            \\
            & r = \sqrt{D^2 + R_0^2}
            \\
            & \versor{r} = \nnorm{\versor{r}} \sqb{\cos(\theta) \, \iVer + \sin(\theta) \, \jVer}
        \end{aligned}
        \right.
    \end{equation*}

    Además, por geometría y simetría:
    \begin{equation*}
        \cos(\theta) = \frac{D}{r} \implies \versor{r} = \frac{D}{r} \, \iVer
    \end{equation*}

    Obteniendo
    \begin{equation*}
        \vec{E} = \frac{\cstcoulomb \, D \, Q}{\bb{D^2 + R_0^2}^{\frac{3}{2}}} \, \iVer
    \end{equation*}
\end{mdframed}

\begin{mdframed}[style=MyFrame2]
    \begin{example}
    \end{example}
    \cusTi{Campo eléctrico de un disco}
    \begin{formatI}
        Calcular el campo eléctrico en el punto $\vec{x}_0$ generado por un disco aislante con densidad de carga superficial uniforme.
    \end{formatI}

    \begin{center}
        \def\svgwidth{\linewidth}
        \input{./images/elec-disco.pdf_tex}
    \end{center}

    Observar que $r$ varía axialmente a medida que se recorre el disco.
    \begin{equation*}
        \vec{E} = \cstcoulomb \iint \frac{\sigma \, \dif A}{r^2} \, \versor{r}
    \end{equation*}

    Cambiando a coordenadas polares:
    \begin{gather*}
        \fx{T}{R,\phi} = (y,z) = \sqb{R \cos(\phi) &,& R \sin(\phi)}
        \\
        \det \bb{\fx{J}{T}} = \cos(\phi) \, R \cos(\phi) - \sin(\phi) \bb{-R \sin(\phi)}
        \\
        \det \bb{\fx{J}{T}} = R \implies
        \\
        \dif A = R \, \dif \phi \, \dif R
    \end{gather*}

    Se puede reescribir la integral de campo eléctrico como sigue:
    \begin{equation*}
        \vec{E} = \cstcoulomb \iint \frac{\sigma \, R \, \dif \phi \, \dif R}{r^2} \, \versor{r}
    \end{equation*}

    Donde:
    \begin{equation*}
        \left\{
        \begin{aligned}
            & r = \sqrt{D^2 + R^2}
            \\
            & \versor{r} = \nnorm{\versor{r}} \sqb{\cos(\theta) \, \iVer + \sin(\theta) \, \jVer}
        \end{aligned}
        \right.
    \end{equation*}

    Además, por geometría y simetría:
    \begin{equation*}
        \cos(\theta) = \frac{D}{r} \implies \versor{r} = \frac{D}{r} \, \iVer
    \end{equation*}
    
    Reemplazando $\versor{r}$ y $r$ nuevamente se obtiene:
    \begin{align*}
        \vec{E} &= \cstcoulomb \, \sigma \, D \iint \frac{R \, \dif \phi \, \dif R}{\bb{D^2 + R^2}^{\frac{3}{2}}} \, \iVer
        \\[1ex]
        &= \cstcoulomb \, \sigma \, D \int_0^{2\pi} \dif \phi \, \int_0^{R_0} \frac{R \, \dif R}{\bb{D^2 + R^2}^{\frac{3}{2}}} \, \iVer
        \\[1ex]
        &= 2 \, \pi \, \cstcoulomb \, \sigma \, D \int_0^{R_0} \frac{R \, \dif R}{\bb{D^2 + R^2}^{\frac{3}{2}}} \, \iVer
    \end{align*}
\end{mdframed}

\begin{mdframed}[style=MyFrame2]
    \begin{example}
        \label{eg:elec-placa}
    \end{example}
    \cusTi{Campo eléctrico de una placa infinita}
    \begin{formatI}
        Calcular el campo eléctrico en el punto $\vec{x}_0$ generado por una placa infinita aislante con densidad de carga superficial uniforme.
    \end{formatI}
    \begin{center}
        \def\svgwidth{\linewidth}
        \input{./images/elec-placa.pdf_tex}
    \end{center}
    Una placa infinita es equivalente a un disco infinito.
    Es posible hacer el mismo razonamiento para el disco y hacer tender el radio a infinito.
    Considerando $R_0 \to \infty$ en el resultado obtenido anteriormente:
    \begin{equation*}
        \vec{E} = 2 \, \pi \, \cstcoulomb \, \sigma \, D \int_0^\infty \frac{R \, \dif R}{\bb{D^2 + R^2}^{\frac{3}{2}}} \, \iVer
    \end{equation*}
    Sea $u = D^2 + R^2$ tal que si $R \to \infty \implies u \to \infty$ y
    \begin{equation*}
        \frac{\dif u}{\dif R} = 2 \, R \implies \dif R = \frac{\dif u}{2 \, R}
    \end{equation*}

    De manera que la integral de campo es
    \begin{align*}
        \vec{E} &= \pi \, \cstcoulomb \, \sigma \, D \int_0^\infty \frac{\dif u}{u^{\frac{3}{2}}} \, \iVer
        \\[1ex]
        &= \pi \, \cstcoulomb \, \sigma \, D \, \barrow{\frac{1}{-\frac{1}{2} \sqrt{u}}}{u=0}{u=\infty}
        \\[1ex]
        &= 2 \, \pi \, \cstcoulomb \, \sigma \, D \, \barrow{\frac{1}{\sqrt{D^2 + R^2}}}{R=\infty}{R=0}
        \\
        &= 2 \, \pi \, \cstcoulomb \, \sigma
        \\
        &= \frac{\sigma}{2 \, \epsilon_0}
    \end{align*}

    Observar que el resultado es independiente de $D$.
    Esto implica que la dirección y magnitud del campo es constante para todo $\vec{x}$.
    
    Este resultado se verificará mediante otro método en el ejemplo \ref{eg:gauss-placa}.
\end{mdframed}

\begin{mdframed}[style=MyFrame2]
    \begin{example}
    \end{example}
    \cusTi{Campo eléctrico de un cilindro}
    \begin{formatI}
        Calcular el campo eléctrico en el punto $\vec{x}_0$ generado por un cilindro aislante con densidad de carga superficial uniforme. Considerar el caso en que el cilindro es hueco y el caso en que es sólido.
    \end{formatI}
    \begin{center}
        \def\svgwidth{\linewidth}
        \input{./images/elec-cilindro.pdf_tex}
    \end{center}

    El campo eléctrico para una carga contínua es
    \begin{equation*}
        \vec{E} = \cstcoulomb \int \frac{\dif q}{r^2} \, \versor{r}
    \end{equation*}

    \concept{Caso de cilindro hueco (sin tapas):}

    El diferencial de carga está dado por
    \begin{equation*}
        \dif q = \sigma \, \dif A
        = \sigma \, \dif s \, \dif x
        = \sigma \, R_0 \, \dif \phi \, \dif x
    \end{equation*}

    Por simetría axial el versor director es
    \begin{equation*}
        \versor{r} = \cos(\theta) \, \iVer
        = \frac{L+D-x}{r} \, \iVer
    \end{equation*}

    Y la distancia $r$ es
    \begin{equation*}
        r = \sqrt{R_0^2 + \bb{L+D-x}^2}
    \end{equation*}

    Con lo cual, la integral de campo eléctrico queda
    \begin{align*}
        \vec{E} &= \cstcoulomb \iint \frac{\sigma \, R_0 \, \dif \phi \, \dif x}{r^2} \, \frac{L+D-x}{r} \, \iVer
        \\[1em]
        &= \cstcoulomb \, \sigma \, R_0 \iint \frac{L+D-x}{r^3} \dif \phi \, \dif x \, \iVer
        \\[1em]
        &= 2 \, \pi \, \cstcoulomb \, \sigma \, R_0 \int_0^L \frac{L+D-x}{\sqb{R_0^2 + \bb{L+D-x}^2}^{\frac{3}{2}}} \dif x \, \iVer
    \end{align*}

    \concept{Caso de cilindro sólido:}

    El diferencial de carga está dado por
    \begin{equation*}
        \dif q = \rho \, \dif V
        = \rho \, \pi \, R_0^2 \, \dif x
    \end{equation*}

    Por simetría axial el versor director es
    \begin{equation*}
        \versor{r} = \cos(\theta) \, \iVer
        = \frac{L+D-x}{r} \, \iVer
    \end{equation*}

    Y la distancia $r$ es
    \begin{equation*}
        r = \sqrt{R_0^2 + \bb{L+D-x}^2}
    \end{equation*}

    Con lo cual, la integral de campo eléctrico queda
    \begin{align*}
        \vec{E} &= \cstcoulomb \int \frac{\rho \, \pi \, R_0^2 \, \dif x}{r^2} \, \frac{L+D-x}{r} \, \iVer
        \\[1em]
        &= \pi \, \cstcoulomb \, \rho \, R_0^2 \int \frac{L+D-x}{r^3} \dif x \, \iVer
        \\[1em]
        &= \pi \, \cstcoulomb \, \rho \, R_0^2 \int_0^L \frac{L+D-x}{\sqb{R_0^2 + \bb{L+D-x}^2}^{\frac{3}{2}}} \dif x \, \iVer
    \end{align*}
\end{mdframed}

\begin{mdframed}[style=MyFrame2]
    \begin{example}
    \end{example}
    \cusTi{Campo eléctrico de un cono}
    \begin{formatI}
        Calcular el campo eléctrico en el punto $\vec{x}_0$ generado por un cono aislante con densidad de carga superficial uniforme. Considerar el caso en que el cono es hueco y el caso en que es sólido.
    \end{formatI}

    \begin{center}
        \def\svgwidth{\linewidth}
        \input{./images/elec-cono.pdf_tex}
    \end{center}

    El campo eléctrico para una carga contínua es
    \begin{equation*}
        \vec{E} = \cstcoulomb \int \frac{\dif q}{r^2} \, \versor{r}
    \end{equation*}

    \concept{Caso de cono hueco (sin tapa):}

    El diferencial de carga está dado por
    \begin{equation*}
        \dif q = \sigma \, \dif A
        = \sigma \, \dif s \, \dif x
        = \sigma \, \fx{R}{x} \, \dif \phi \, \dif x
    \end{equation*}

    Por simetría axial el versor director es
    \begin{equation*}
        \versor{r} = \cos(\theta) \, \iVer
        = \frac{D-x}{r} \, \iVer
    \end{equation*}

    Y la distancia $r$ es
    \begin{equation*}
        r = \sqrt{\bb{D-x}^2 + \sqb{\fx{R}{x}}^2}
    \end{equation*}

    Al tratarse de la pendiente de la recta del cono en corte $y=0$, el radio varía según
    \begin{equation*}
        \fx{R}{x} = \frac{R_0}{L} \, x
    \end{equation*}

    Con lo cual, la integral de campo eléctrico queda
    \begin{align*}
        \vec{E} &= \cstcoulomb \iint \frac{\sigma \, \fx{R}{x} \, \dif \phi \, \dif x}{\bb{D-x}^2 + \sqb{\fx{R}{x}}^2} \, \frac{D-x}{r} \, \iVer
        \\[1em]
        &= \cstcoulomb \, \sigma \, \frac{R_0}{L} \iint \frac{x \bb{D-x}}{\sqb{\bb{D-x}^2 + \bb{\frac{R_0}{L} \, x}^2}^{\frac{3}{2}}} \, \dif \phi \, \dif x \, \iVer
        \\[1em]
        &= 2 \, \pi \, \cstcoulomb \, \sigma \, \frac{R_0}{L} \int_0^L \frac{x \bb{D-x}}{\sqb{\bb{D-x}^2 + \bb{\frac{R_0}{L} \, x}^2}^{\frac{3}{2}}} \, \dif x \, \iVer
    \end{align*}

    \concept{Caso de cono sólido:}

    \begin{equation*}
        \dif q = \rho \, \dif V = \rho \, \pi \, \sqb{\fx{R}{x}}^2 \, \dif x
    \end{equation*}

    Por simetría axial el versor director es
    \begin{equation*}
        \versor{r} = \cos(\theta) \, \iVer
        = \frac{D-x}{r} \, \iVer
    \end{equation*}

    Y la distancia $r$ es
    \begin{equation*}
        r = \sqrt{\bb{D-x}^2 + \sqb{\fx{R}{x}}^2}
    \end{equation*}

    Al tratarse de la pendiente de la recta del cono en corte $y=0$, el radio varía según
    \begin{equation*}
        \fx{R}{x} = \frac{R_0}{L} \, x
    \end{equation*}

    Con lo cual, la integral de campo eléctrico queda
    \begin{align*}
        \vec{E} &= \cstcoulomb \int \frac{\rho \, \pi \, \sqb{\fx{R}{x}}^2 \, \dif x}{r^2} \, \frac{D-x}{r} \, \iVer
        \\[1em]
        &= \pi \, \cstcoulomb \, \rho \int \frac{\sqb{\fx{R}{x}}^2 \bb{D-x}}{r^3} \, \dif x \, \iVer
        \\[1em]
        &= \pi \, \cstcoulomb \, \rho \bb{\frac{R_0}{L}}^2 \int_0^L \frac{x^2 \bb{D-x}}{\sqb{\bb{D-x}^2 + \bb{\frac{R_0}{L} \, x}^2}^{\frac{3}{2}}} \, \dif x \, \iVer
    \end{align*}
\end{mdframed}


\section{Ley de Gauss}

El flujo eléctrico ($\Phi$) de una carga eléctrica encerrada por una superficie ($S$) está dado por el campo eléctrico ($\Vec{E}$) según
\begin{equation*}
    \Phi = \iint_S \Vec{E} \cdot \dif \Vec{S} = \iint \Vec{E} \bb{\Vec{s}(u,v)} \versor{n} \, \dif S
\end{equation*}

La Ley de Gauss establece que el flujo no depende de la ubicación de la carga dentro de una superficie cerrada. Esto se debe a que el campo eléctrico es inversamente proporcional a la distancia mientras que el área es directamente proporcional.
Así, podemos reducir cualquier caso de estudio a una superficie esférica con la carga ubicada en el centro:
\begin{equation*}
    \Phi = E \, A = \frac{\cstcoulomb \, \sub{q}{in}}{r^2} \, 4 \, \pi \, r^2 = 4 \, \pi \, \cstcoulomb \, \sub{q}{in} = \frac{\sub{q}{in}}{\epsilon_0}
\end{equation*}

Si el campo eléctrico tiene magnitud $E$ constante y dirección normal a una superficie de área $A$ entonces se puede calcular la magnitud del campo en un punto de la superficie.
\begin{equation*}
    \Phi = E \, \oiint_S \dif S = E \, A
\end{equation*}

\begin{mdframed}[style=MyFrame2]
    \begin{example}
    \end{example}
    \cusTi{Simetría cilíndrica}
    \begin{formatI}
        Calcular el campo eléctrico en el punto $\vec{x}_0$ generado por una barra infinita aislante con densidad de carga lineal uniforme.
    \end{formatI}
    \begin{center}
        \def\svgwidth{0.6\linewidth}
        \input{./images/gauss-barra.pdf_tex}
    \end{center}
    \begin{equation*}
        \left\{
        \begin{aligned}
            \Phi &= E \, A_2 = E \, 2 \, \pi \, r \, L
            \\
            \Phi &= \frac{\sub{q}{in}}{\epsilon_0} = \frac{\lambda \, L}{\epsilon_0}
        \end{aligned}
        \right.
    \end{equation*}
    Luego:
    \begin{equation*}
        E = \frac{\lambda}{2 \, \pi r \, \epsilon_0}
    \end{equation*}
\end{mdframed}

\begin{mdframed}[style=MyFrame2]
    \begin{example}
        \label{eg:gauss-placa}
    \end{example}
    \cusTi{Simetría bilateral}
    \begin{formatI}
        Calcular el campo eléctrico en el punto $\vec{x}_0$ generado por una placa infinita aislante con densidad de carga superficial uniforme.
    \end{formatI}
    \begin{center}
        \def\svgwidth{\linewidth}
        \input{./images/gauss-placa.pdf_tex}
    \end{center}
    \begin{equation*}
        \left\{
        \begin{aligned}
            \Phi &= E \, \sub{A}{tot} = E \bb{A_1+A_2} = 2 \, E \, A
            \\
            \Phi &= \frac{\sub{q}{in}}{\epsilon_0} = \frac{\sigma \, A}{\epsilon_0}
        \end{aligned}
        \right.
    \end{equation*}
    Luego:
    \begin{equation*}
        E = \frac{\sigma}{2 \, \epsilon_0}
    \end{equation*}
    Obteniendo el mismo resultado que en el ejemplo \ref{eg:elec-placa} independientemente de $D$.
\end{mdframed}

\begin{mdframed}[style=MyFrame2]
    \begin{example}
    \end{example}
    \cusTi{Simetría esférica}
    \begin{formatI}
        Calcular el campo eléctrico en el punto $\vec{x}_0$ generado por una esfera aislante de radio $R$ que tiene una densidad de carga volumétrica uniforme.
    \end{formatI}
    \begin{center}
        \def\svgwidth{0.7\linewidth}
        \input{./images/gauss-esfera.pdf_tex}
    \end{center}
    \begin{equation*}
        \left\{
        \begin{aligned}
            \Phi &= E \, A = E \, 4 \pi \, r^2
            \\
            \Phi &= \frac{\sub{q}{in}}{\epsilon_0}
        \end{aligned}
        \right.
    \end{equation*}
    Luego:
    \begin{equation*}
        E = \frac{\sub{q}{in}}{4 \, \pi \, \epsilon_0 \, r^2}
    \end{equation*}
    Según el volumen de carga encerrado $V$ por la superficie gaussiana, se van a obtener distintos valores para la magnitud del campo.
    La carga encerrada depende del radio $r$, ya que está dada por
    \begin{equation*}
        \fx{\sub{q}{in}}{r} = \rho \, \fx{V}{r}
    \end{equation*}
    Pero el volumen encerrado no puede ser mayor que $\fx{V}{R}$ pudiendo distinguir dos casos:
    \begin{equation*}
        \left\{
        \begin{aligned}
            r<R & \implies q_1 = \rho \, \fx{V}{r} = \frac{Q}{\frac{4 \, \pi \, R^3}{3}} \, \frac{4 \, \pi \, r^3}{3} = \frac{Q \, r^3}{R^3}
            \\
            r>R & \implies q_2 = \rho \, \fx{V}{R} = \frac{Q}{\frac{4 \, \pi \, R^3}{3}} \, \frac{4 \, \pi \, R^3}{3} = Q
        \end{aligned}
        \right.
    \end{equation*}
    Obteniendo las siguientes expresiones para el campo eléctrico al reemplazar $q_1$ o $q_2$:
        \begin{equation}
            r<R \implies E = \frac{Q \, r}{4 \, \pi \, \epsilon_0 \, R^3}
            \label{eqn:esferaInt}
        \end{equation}
        \begin{equation}
            r>R \implies E = \frac{Q}{4 \, \pi \, \epsilon_0 \, r^2}
            \label{eqn:esferaExt}
        \end{equation}
    \begin{center}
        \def\svgwidth{0.7\linewidth}
        \input{./images/gauss-esfera-2.pdf_tex}
    \end{center}
\end{mdframed}


\subsection{Equilibrio electrostático}

Se dice que un conductor esta en equilibrio electrostático cuando no tiene electrones en movimiento.

En otras palabras, se trata de un conductor por el que no pasa corriente.
Los conductores que cumplan esta condición, tienen las siguientes propiedades:

\begin{itemize}
\item El campo eléctrico es nulo en el interior del conductor, ya sea este hueco o sólido.

\item Si es un conductor con carga, está se acumula en la superficie.

\item El campo es ortogonal a la superficie del conductor.

\item La densidad de carga superficial es máxima donde la curvatura sea mínima.
\end{itemize}

\begin{mdframed}[style=MyFrame2]
    \begin{example}
    \end{example}
    \begin{formatI}
        Se tiene una esfera aislante de radio $r_0$ con carga $q_1 = Q$ que se encuenta en el centro de un cascarón conductor de ancho $\Delta r = r_2 - r_1$ que tiene $q_2 = -2\,Q$ de carga.
        Se quiere calcular el campo eléctrico en todo el espacio en función de la distancia al centro de la esfera.
    \end{formatI}
    \begin{center}
        \def\svgwidth{0.7\linewidth}
        \input{./images/eq-electrostatico.pdf_tex}
    \end{center}
    \begin{itemize}
        \item Para $r<r_0$ el cascarón no afecta, por lo que el campo está dado según la ecuación \ref{eqn:esferaInt}.
        \item Para $r_0<r<r_1$ el cascarón no afecta, por lo que el campo está dado según la ecuación \ref{eqn:esferaExt}.
        \item Para $r_1<r<r_2$ el campo eléctrico es nulo por tratarse del interior de un conductor.
        \item Para $r_2<r$ el campo es $E = \frac{-Q}{4 \, \pi \, \epsilon_0 \, r^2}$ dado que la carga neta que encierra una superficie gaussiana es $\sub{q}{in} = q_1 + q_2 = -Q$.
    \end{itemize}
\end{mdframed}


\section{Potencial eléctrico}

Si dos cargas son del mismo signo, el medio tiene que hacer trabajo para acercarlas, ganando el sistema energía potencial eléctrica.
Si dos cargas son de signo opuesto, el sistema gana energía potencial cuando estas son alejadas, ya que el medio tiene que hacer trabajo para alejarlas.
Así, se establece por convención que el trabajo es positivo cuando el sistema gana energía.
\begin{equation*}
    W \gtrless 0 \iff {\sub{E}{pot}}_F \gtrless {\sub{E}{pot}}_0 \iff \Delta \sub{E}{pot} \gtrless 0
\end{equation*}

Se tiene una carga $q_1 > 0$ fija en un punto.
Se coloca una segunda carga $q_2 < 0$ a una distancia infinitamente grande
En principio $q_2$ no se vería afectada por el campo eléctrico de $q_1$, ya que en el infinito sería nulo.
Ahora bien, si se la perturba de la posición inicial, partiría con cierta velocidad inicial para acelerarse en dirección hacia $q_1$.
La fuerza causante de esta aceleración sería la fuerza eléctrica ejercida por $q_1$.

No obstante, se aplica sobre $q_2$ una fuerza externa de igual magnitud y sentido contrario a la fuerza eléctrica.
Así, $q_2$ se mueve a velocidad constante en dirección hacia $q_1$ tal que $\Delta \sub{E}{cin} = 0$.

El trabajo de la fuerza externa cuantifica la energía que tiene $q_2$ por estar en presencia del campo producido por $q_1$, energía que potencialmente podría convertirse en energía cinética si no hubiese fuerza externa.
Pero el trabajo de la fuerza externa (no conservativa) es igual a menos el trabajo de la fuerza eléctrica (conservativa).
Esto es, renombrando a $q_2$ como $q_0$:
\begin{align*}
    \Delta \sub{E}{pot} &= - \sub{W}{con}
    \\
    &= - \int_C \sub{\Vec{F}}{ele} \cdot \dif \Vec{s}
    \\
    &= - q_0 \int_C \Vec{E} \cdot \dif \Vec{s}
\end{align*}

Dado que $q_0 < 0$ y $-\sub{W}{con} = \sub{W}{no con} < 0$ es posible mantener el menos en la ecuación y trabajar con valores absolutos ya que los negativos se cancelan entre sí.
Es decir, si se hubiese tratado de una carga de prueba positiva, el trabajo de la fuerza externa para llevar $q_2$ en dirección hacia $q_1$ hubiese sido positivo también.

Finalmente, se define la diferencia de potencial eléctrico $(\Delta v)$ dividiendo por $q_0$ ambos miembros.

\begin{mdframed}[style=MyFrame1]
    \begin{defn}
        \label{defn:potEle}
    \end{defn}
    \cusTi{Potencial eléctrico}
    \begin{equation*}
        \Delta \voltage = - \int_C \vec{E} \cdot \dif \vec{s}
    \end{equation*}
\end{mdframed}

A partir de la definición anterior, se deduce el potencial eléctrico para una carga puntual:
\begin{align*}
    \Delta \voltage &= - \int_C \frac{\cstcoulomb \, q}{r^2} \versor{r} \cdot \dif \Vec{s}
    \\
    &= - \int_C \frac{\cstcoulomb \, q}{r^2} \nnorm{\versor{r}} \, \nnorm{\dif \Vec{s}} \, \cos(\theta)
    \\
    &= - \int_{r_0}^{r_1} \frac{\cstcoulomb \, q}{r^2} \dif r
    \\
    &= \barrow{\frac{\cstcoulomb \, q}{r}}{r_0}{r_1}
\end{align*}

Pudiendo luego generalizar para $N$ cargas puntuales:
\begin{equation}
    \Delta \voltage = \cstcoulomb \sum_{\ith=1}^\nth \frac{q_\ith}{r_\ith}
\end{equation}

Y tomando el límite cuando $\nth\to\infty$ en la ecuación anterior, se define el potencial en un punto del espacio dado por una carga contínua.

\begin{mdframed}[style=MyFrame1]
    \begin{defn}
        \label{defn:potCargaCont}
    \end{defn}
    \cusTi{Potencial eléctrico de carga contínua}
    \begin{equation*}
        \Delta \voltage = \cstcoulomb \int \frac{\dif q}{r}
    \end{equation*}
\end{mdframed}

La definición \ref{defn:potEle} está dada para cargas puntuales.
No confundir la integral involucrada con la integral de la definición \ref{defn:potCargaCont}.
La primera implica una integral de tipo 2 para el trabajo, mientras que la segunda es una integral de Riemann para sumar el aporte de infinitas cargas puntuales.

\begin{mdframed}[style=MyFrame2]
    \begin{example}
    \end{example}
    \cusTi{Potencial en campo uniforme}
    \cusTe{Se quiere calcular el potencial eléctrico o bien entre dos placas paralelas con carga opuesta o bien para una única placa infinita con cierta carga.}
    Partiendo de la definición \ref{defn:potEle} se calcula el potencial.
    Esto es, la energía cinética que ganaría una carga de prueba en ir desde una placa hacia la otra, o en recorrer cierta distancia $\Delta x$ por la fuerza eléctrica ejercida por una placa infinita.
    \begin{align*}
        \Delta \voltage &= - E \int_{x_0}^{x_1} \dif x
        \\
        &= - E \, \Delta x
    \end{align*}
\end{mdframed}

\begin{mdframed}[style=MyFrame2]
    \begin{example}
    \end{example}
    \cusTi{Potencial de un dipolo}
    \begin{formatI}
        Calcular el potencial eléctrico en los puntos $\vec{x}_1$ y $\vec{x}_2$ que es generado por un dipolo de cargas puntuales $q_1=-q_2=Q$.
    \end{formatI}
    \begin{center}
        \def\svgwidth{0.7\linewidth}
        \input{./images/pot-dipolo.pdf_tex}
    \end{center}
    En el punto $\vec{x}_1$ el potencial es:
    \begin{equation*}
        \Delta \voltage = \cstcoulomb \bb{\frac{q_1}{\sqrt{R^2 + y_0^2}} + \frac{q_2}{\sqrt{R^2 + y_0^2}}} = 0
    \end{equation*}

    En el punto $\vec{x}_2$ el potencial es:
    \begin{equation*}
        \Delta \voltage = \cstcoulomb \bb{\frac{q_1}{x_0 + R} + \frac{q_2}{x_0 - R}}
    \end{equation*}
\end{mdframed}

\begin{mdframed}[style=MyFrame2]
    \begin{example}
    \end{example}
    \cusTi{Potencial de una barra}
    \begin{formatI}
        Calcular el potencial eléctrico en el punto $\vec{x}_0$ generado por una barra con densidad de carga lineal uniforme.
    \end{formatI}
    \begin{center}
        \def\svgwidth{\linewidth}
        \input{./images/pot-barra.pdf_tex}
    \end{center}
    Para una distribución contínua, el potencial es el aporte de cada diferencial de carga:
    \begin{equation*}
        \Delta \voltage = \cstcoulomb \int \frac{\dif q}{r}
    \end{equation*}
    Donde:
    \begin{equation*}
        \left\{
        \begin{aligned}
            \dif q &= \lambda \, \dif x
            \\
            r &= \sqrt{x^2+H^2}
        \end{aligned}
        \right.
    \end{equation*}
    Luego:
    \begin{equation*}
        \Delta \voltage = \cstcoulomb \, \lambda \int_D^{D+L} \frac{1}{\sqrt{x^2+H^2}} \, \dif x
    \end{equation*}
\end{mdframed}

\begin{mdframed}[style=MyFrame2]
    \begin{example}
    \end{example}
    \cusTi{Potencial de un anillo}
    \begin{formatI}
        Calcular el potencial eléctrico en el punto $\vec{x}_0$ generado por un anillo aislante con carga $Q$ uniformemente distribuída.
    \end{formatI}
    \begin{center}
        \def\svgwidth{\linewidth}
        \input{./images/pot-anillo.pdf_tex}
    \end{center}
    Para una distribución contínua, el potencial es el aporte de cada diferencial de carga:
    \begin{equation*}
        \Delta \voltage = \cstcoulomb \int \frac{\dif q}{r}
    \end{equation*}
    Donde:
    \begin{equation*}
        r = \sqrt{D^2 + R_0^2}
    \end{equation*}
    Luego:
    \begin{equation*}
        \Delta \voltage = \frac{\cstcoulomb}{\sqrt{D^2 + R_0^2}} \int \dif q = \frac{\cstcoulomb \, Q}{\sqrt{D^2 + R_0^2}}
    \end{equation*}
\end{mdframed}

\begin{mdframed}[style=MyFrame2]
    \begin{example}
    \end{example}
    \cusTi{Potencial de un disco}
    \begin{formatI}
        Calcular el potencial eléctrico en el punto $\vec{x}_0$ generado por un disco con densidad de carga superficial uniforme.
    \end{formatI}
    \begin{center}
        \def\svgwidth{\linewidth}
        \input{./images/pot-disco.pdf_tex}
    \end{center}
    Para una distribución contínua, el potencial es el aporte de cada diferencial de carga:
    \begin{equation*}
        \Delta \voltage = \cstcoulomb \int \frac{\dif q}{r}
    \end{equation*}
    Donde:
    \begin{equation*}
        \left\{
        \begin{aligned}
            \dif q &= \sigma \, \dif y \, \dif z = \sigma \, R \, \dif R \, \dif \phi
            \\
            r &= \sqrt{D^2+R^2}
        \end{aligned}
        \right.
    \end{equation*}
    Luego:
    \begin{align*}
        \Delta \voltage &= \cstcoulomb \, \sigma \, \int_0^{2\pi} \dif \phi \int_0^{R_0} \frac{R}{\sqrt{D^2+R^2}} \, \dif R
        \\[1ex]
        &= 2 \, \pi \, \cstcoulomb \, \sigma \bb{\sqrt{R_0^2 + D^2} - D}
    \end{align*}
\end{mdframed}


\subsection{Generador Van de Graaff}

\begin{center}
    \def\svgwidth{0.8\linewidth}
    \input{./images/pot-van-de-graaff.pdf_tex}
\end{center}


\subsection{Relación entre campo y potencial}

La diferencia de potencial eléctrico, es conocida como tensión eléctrica, voltaje, potencial eléctrico o simplemente potencial.

De la definición \ref{defn:potEle} se deduce que el diferencial de potencial eléctrico es
\begin{equation*}
    \dif \voltage = - \Vec{E} \, \dif \Vec{s}
\end{equation*}

Lo cual implica $\Vec{E} = - \grad \voltage$ que en una dimensión es:
\begin{equation*}
    E(x) = - \frac{\dif}{\dif x} \voltage(x)
\end{equation*}

La siguiente imagen representa una esfera conductora con carga positiva. Se observa que a medida que aumenta la distancia $(r)$, el campo $(E)$ disminuye.
La gráfica de $\voltage(r)$ es decreciente, y siempre tiene pendiente negativa.
La pendiente de $\voltage(r)$ es la derivada con respecto a la distancia que, por definición, es el campo eléctrico.

\begin{center}
    \def\svgwidth{0.8\linewidth}
    \input{./images/pot-ele-esfera-cond.pdf_tex}
\end{center}

El signo negativo se deduce por ser el campo eléctrico positivo si $r > R$ y negativo si $r < -R$ y la pendiente de $\voltage(r)$ negativa y positiva respectivamente.
Además, se puede observar que si la distancia es $-R < r < R$, el potencial eléctrico $\voltage(r)$ es constante con lo cual el campo es nulo ya que es la derivada con respecto de la distancia.


\section{Capacitancia}

Un recipiente con un volumen $(V)$ mayor va a tener más capacidad de almacenar gas. Esto va a tener ciertas implicancias sobre la masa $(m)$ y la presión $(P)$.
\begin{equation*}
    V_1 > V_2 \Rightarrow
    \left\{
    \begin{aligned}
        m_1 = m_2 & \Rightarrow P_1 < P_2
        \\
        P_1 = P_2 & \Rightarrow m_1 > m_2
    \end{aligned}
    \right.
\end{equation*}

Si se define la capacidad del recipiente como $C= \sfrac{m}{P}$, a partir de las implicaciones anteriores se puede concluir que el recipiente de mayor volumen es efectivamente el de mayor capacidad.
\begin{gather*}
    \left\{
    \begin{aligned}
        P_1 < P_2 & \Rightarrow \frac{m}{P_1} > \frac{m}{P_2} \Rightarrow C_1 > C_2
        \\[1ex]
        m_1 > m_2 & \Rightarrow \frac{m_1}{P} > \frac{m_2}{P} \Rightarrow C_1 > C_2
    \end{aligned}
    \right.
    \\[1em]
    C_1 > C_2 \Rightarrow V_1 > V_2
\end{gather*}

La capacitancia o capacidad eléctrica es la cantidad de carga $(q)$ por unidad de tensión $(\Delta \voltage)$ que un capacitor o condensador puede almacenar.

\begin{mdframed}[style=MyFrame1]
    \begin{defn}
    \end{defn}
    \cusTi{Capacitancia}
    \begin{equation*}
        C = \frac{q}{\Delta \voltage}
    \end{equation*}
\end{mdframed}

\begin{mdframed}[style=MyFrame2]
    \begin{example}
    \end{example}
    \cusTi{Capacitor de placas}
    \begin{formatI}
        Capacitancia que generan dos placas paralelas, considerando que el campo eléctrico entre ellas es uniforme.
    \end{formatI}
    La capacidad está dada por
    \begin{equation*}
        C = \frac{q}{E \, \Delta x} = \frac{q}{\frac{\sigma}{\epsilon_0}\Delta x} = \frac{q}{\tfrac{q}{A \, \epsilon_0}\Delta x}
    \end{equation*}

    Obteniendo así
    \begin{equation*}
        C = \frac{\epsilon_0 A}{\Delta x}
    \end{equation*}
\end{mdframed}

\begin{mdframed}[style=MyFrame2]
    \begin{example}
    \end{example}
    \cusTi{Capacitor esférico}
    \begin{formatI}
        Capacitancia que generan o bien dos esferas concéntricas, o bien una única esfera como un caso particular en la que la externa tiene radio infinito.
    \end{formatI}
    El potencial eléctrico entre las esferas (en valor absoluto) está dado por:
    \begin{align*}
        \Delta \voltage &= \int_C \vec{E} \cdot \dif \vec{s}
        \\[1ex]
        &= \int_{r_1}^{r_2} \frac{\cstcoulomb \, q}{r^2} \dif r
        \\
        &= \cstcoulomb \, q \bb{\frac{1}{r_1} - \frac{1}{r_2}}
    \end{align*}

    Así:
    \begin{align*}
        C &= \frac{q}{\Delta \voltage}
        \\
        &= \frac{1}{\cstcoulomb \bb{\frac{1}{r_1} - \frac{1}{r_2}}}
    \end{align*}

    Obteniendo, para dos esferas concéntricas
    \begin{equation*}
        C = \frac{r_1 \, r_2}{\cstcoulomb \bb{r_2 - r_1}}
    \end{equation*}

    O bien, si $r_2 \to \infty$, se tiene para una única esfera
    \begin{equation*}
        C = \frac{r_1}{\cstcoulomb} = 4 \, \pi \, \epsilon_0 \, r_1
    \end{equation*}
\end{mdframed}

Para calcular la energía almacenada en un capacitor se hace:
\begin{gather*}
    \frac{\dif W}{\dif q} = \frac{\dif q}{\dif q} \, \Delta \voltage = \Delta \voltage
    \\
    \dif W = \Delta \voltage \, \dif q = \frac{q}{C} \, \dif q
    \\
    \int \dif W = \int \frac{q}{C} \, \dif q
    \\
    W = \barrow{\frac{q^2}{2C}}{0}{q_1}
    \\
    \sub{E}{pot} = \frac{q^2}{2C} = \frac{q \, \Delta \voltage}{2}
\end{gather*}

\begin{mdframed}[style=MyFrame1]
    \begin{prop}
    \end{prop}
    \cusTi{Energía almacenada en un capacitor}
    \begin{equation*}
        \sub{E}{pot} = \frac{C \bb{\Delta \voltage}^2}{2}
    \end{equation*}
\end{mdframed}

\begin{mdframed}[style=MyFrame2]
    \begin{example}
    \end{example}
    \cusTi{Energía almacenada en placas paralelas}

    El potencial eléctrico es
    \begin{equation*}
        \Delta \voltage = E \, \Delta x
    \end{equation*}
    y la capacitancia
    \begin{equation*}
        C = \tfrac{\epsilon_0 \, A}{\Delta x}
    \end{equation*}
    obteniendo
    \begin{equation*}
        \sub{E}{pot} = \tfrac{1}{2} \, \tfrac{\epsilon_0 \, A}{\Delta x} \bb{E \, \Delta x}^2
    \end{equation*}
    y si $V_0 = A \, \Delta x$ es el volumen se tiene
    \begin{equation*}
        \frac{\sub{E}{pot}}{V_0} = \frac{\epsilon_0 \, E^2}{2}
    \end{equation*}
\end{mdframed}

El campo eléctrico máximo que se puede dar en un capacitor está dado por la resistencia dieléctrica. Si la magnitud del campo es mayor que esta, esto es, entonces el dieléctrico pasa a ser conductor. Por lo tanto, para un dieléctrico se cumple:
\begin{equation*}
    E < \epsilon
\end{equation*}

La capacitancia está definida para el vacío. Si entre los conductores de un capacitor en vez de vacío hay un material dieléctrico, la tensión va a estar dada por
\begin{equation*}
    \Delta \voltage = \frac{\Delta V_0}{k}
\end{equation*}

Quedando la capacitancia:
\begin{equation*}
    C = k \, C_0
\end{equation*}


\chapter{Circuitos DC}
\renewcommand{\voltage}{V}
\renewcommand{\current}{I}

\section{Corriente}

Alessandro Volta inventó la pila de corriente continua colocando un electrolito entre dos metales con diferente potencial de extracción.

\begin{mdframed}[style=MyFrame1]
    \begin{defn}
    \end{defn}
    \cusTi{Corriente}
    \cusTe{La intensidad de corriente es la cantidad de cargas por segundo que pasa por un punto en un circuito.}
    \begin{equation*}
        I = \frac{q}{\Delta t}
    \end{equation*}
\end{mdframed}


\section{Ley de Ohm}

Los conductores ideales no oponen resistencia alguna al flujo de corriente. Los buenos conductores practicamente no se oponen al paso de corriente, como el oro o el cobre. Los malos conductores hacen que el flujo de corriente sea lento.

Cuanto más largo sea un mal conductor, más resistencia opondrá al paso de corriente. Cuanto más ancho sea, las cargas tendrán más lugar para fluir:

\begin{equation*}
    \text{Resistencia} \equiv \frac{\text{Resistividad} \times \text{Distancia}}{\text{Area}}
\end{equation*}

Georg Simon Ohm demostró experimentalmente la proporción con la que, para distintos tipos de conductores, una misma tensión genera distintos flujos de corriente. Pero indistintamente del material y topología del conductor, para cada uno se verifica que la relación entre diferentes tensiones con las respectivas corrientes que se generan es constante:

\begin{mdframed}[style=MyFrame1]
    \begin{defn}
    \end{defn}
    \cusTi{Ley de Ohm}
    \cusTe{La tensión es directamente proporcional a la corriente, siendo $R$ el factor de proporcionalidad.}
    \begin{equation*}
        R = \frac{V}{I}
    \end{equation*}
\end{mdframed}


\section{Leyes de Kirchhoff}

\begin{itemize}
\item La ley de Kirchhoff de la Corriente dice que para cada nodo, la suma de las corrientes es nula, considerando la corriente entrante como positiva y saliente como negativa.

\item La Ley de Kirchhoff de la Tensión dice que para cada malla, si esta se recorre en sentido horario la suma de las tensiones de los componentes que se encuentren es nula.
\end{itemize}


\section{Serie y paralelo}


\subsection{Capacitores en paralelo}

\begin{gather*}
    \left\{
    \begin{aligned}
        V &= \norm{V_1} = \norm{V_2}
        \\
        q &= q_1 + q_2
    \end{aligned}
    \right.
    \\[1ex]
    \sub{C}{eq} = \frac{q}{V} = \frac{q_1 + q_2}{V} = \frac{q_1}{V} + \frac{q_2}{V}
\end{gather*}

\begin{mdframed}[style=MyFrame1]
    \begin{prop}
    \end{prop}
    \cusTi{Capacitores en paralelo}
    \begin{equation*}
        \sub{C}{eq} = \sum_{\ith=1}^\nth C_\ith
    \end{equation*}
\end{mdframed}


\subsection{Capacitores en serie}

\begin{gather*}
    \left\{
    \begin{aligned}
        V &= \norm{V_1} + \norm{V_2}
        \\
        q &= q_1 = q_2
    \end{aligned}
    \right.
    \\[1ex]
    \sub{C}{eq} = \frac{q}{V} = \frac{q}{V_1 + V_2}
\end{gather*}

\begin{mdframed}[style=MyFrame1]
    \begin{prop}
    \end{prop}
    \cusTi{Capacitores en serie}
    \begin{equation*}
        \frac{1}{\sub{C}{eq}} = \sum_{\ith=1}^\nth \frac{1}{C_\ith}
    \end{equation*}
\end{mdframed}


\subsection{Resistencias en serie}

\begin{gather*}
    \left\{
    \begin{aligned}
        V &= \norm{V_1} + \norm{V_2}
        \\
        q &= q_1 = q_2
    \end{aligned}
    \right.
    \\[1ex]
    \sub{R}{eq} = \frac{V \, \Delta t}{q} = \frac{\bb{V_1 + V_2} \Delta t}{q} = \frac{\bb{I \, R_1 + I \, R_2}}{I} = R_1 + R_2
\end{gather*}

\begin{mdframed}[style=MyFrame1]
    \begin{prop}
    \end{prop}
    \cusTi{Resistencias en serie}
    \begin{equation*}
        \sub{R}{eq} = \sum_{\ith=1}^\nth R_\ith
    \end{equation*}
\end{mdframed}


\subsection{Resistencias en paralelo}

\begin{gather*}
    \left\{
    \begin{aligned}
        V &= \norm{V_1} = \norm{V_2}
        \\
        q &= q_1 + q_2
    \end{aligned}
    \right.
    \\[1ex]
    \sub{R}{eq} = \frac{V \, \Delta t}{q}
    = \frac{V \, \Delta t}{q_1 + q_2}
    = \frac{V}{I_1 + I_2} = \frac{V}{\dfrac{V}{R_1} + \dfrac{V}{R_2}}
    = \frac{1}{\dfrac{1}{R_1}+\dfrac{1}{R_2}}
\end{gather*}

\begin{mdframed}[style=MyFrame1]
    \begin{prop}
    \end{prop}
    \cusTi{Resistencias en paralelo}
    \begin{equation*}
        \frac{1}{\sub{R}{eq}} = \sum_{\ith=1}^\nth \frac{1}{R_\ith}
    \end{equation*}
\end{mdframed}


\section{Fuente ideal y fuente real}

Las fuentes reales generan una tensión, pero al conectarlas con una resistencia de carga, varian la tensión que generarian de no tener carga.


\subsection{Modelo de fuente de tensión real}

Para simular el efecto de disminución del potencial que generaría conectar un resistor de carga $(R_L)$ a una fuente ideal $(V_F)$, se puede incluir un resistor en serie, llamado resistencia de fuente $(R_S)$.
El modelo de fuente real $(V_S)$ es el conjunto de la fuente ideal y la resistencia de fuente.

La resistencia de fuente va a depender del resistor de carga, ya que queremos que al conectarla, la fuente real tenga un valor que se aproxime al de la fuente ideal. Los esquemas se muestran a continuación:

\begin{multicols}{2}
    \begin{center}
        \def\svgwidth{0.9\linewidth}
        \input{./images/fuente-de-tension-modelo-1.pdf_tex}
    \end{center}
    \begin{center}
        \def\svgwidth{0.9\linewidth}
        \input{./images/fuente-de-tension-modelo-2.pdf_tex}
    \end{center}
\end{multicols}

Nótese que la corriente $(I)$ no va a variar cuando se analiza el circuito para la fuente ideal con respecto del circuito para la fuente real, por tratarse de conexiones en serie:
\begin{equation*}
    \left\{
    \begin{aligned}
        V_F &= V_{R_S} + V_{R_L} = I \bb{R_S + R_L}
        \\
        V_S &= V_{R_L} = I \, R_L
    \end{aligned}
    \right.
\end{equation*}

Al comparar la tensión de salida $(V_S)$ con la tensión $(V_F)$ que tendría la fuente ideal, se pueden sacar conclusiones de cómo deberá ser $R_L$ con respecto de $R_S$, para que $V_S$ se aproxime a $V_F$.
\begin{equation*}
    \frac{V_S}{V_F} = \frac{I \, R_L}{I \bb{R_S + R_L}} = \frac{R_L}{R_S + R_L}
\end{equation*}

Por lo tanto, $V_S \to V_F$ cuando $R_S / R_L \to 0$. Entonces, para que el modelo de fuente real sea una buena aproximación, se tiene que cumplir que $R_L >> R_S$.

\begin{center}
    \def\svgwidth{0.6\linewidth}
    \input{./images/fuente-de-tension-modelo-5.pdf_tex}
\end{center}


\subsection{Modelo $A$ de fuente de corriente}

Una fuente de corriente es un elemento teórico que no existe en la realidad de manera natural. Pero como la tensión es proporcional a la corriente existen fuentes de tensión que funcionan como fuentes de corrientes, bajo ciertas consideraciones.

En la situación anterior, la disminución de tensión en una fuente real va a ser proporcional a una disminución de corriente.
Con este criterio, para simular una fuente de corriente real se puede usar el circuito anterior.
Los esquemas se muestran a continuación:

\begin{multicols}{2}
    \begin{center}
        \def\svgwidth{0.9\linewidth}
        \input{./images/fuente-de-tension-modelo-1.pdf_tex}
    \end{center}
    \begin{center}
        \def\svgwidth{0.9\linewidth}
        \input{./images/fuente-de-tension-modelo-4.pdf_tex}
    \end{center}
\end{multicols}

Nótese que, si bien se trata de conexiones en serie, estamos suponiendo que la corriente va a variar justamente por cómo se comporta el modelo $I_S$ con y sin carga:
\begin{equation*}
    \left\{
    \begin{aligned}
        I_F &= \frac{V_F}{R_S}
        \\[1ex]
        I_S &= \frac{V_F}{R_S + R_L}
    \end{aligned}
    \right.
\end{equation*}

Con el fin de usar este circuito como fuente de corriente, al comparar la corriente de salida $(I_S)$ que entregaría la fuente con la que efectivamente pasa por la fuente ideal $(I_F)$, se tiene que:
\begin{equation*}
    \frac{I_S}{I_F} = \frac{R_S}{R_S + R_L}
\end{equation*}

Es decir, que $I_S \to I_F$ cuando $R_L/R_S \to 0$. Por lo tanto, para que el modelo de fuente de corriente real sea una buena aproximación, se tiene que cumplir que $R_L<<R_S$.

\begin{center}
    \def\svgwidth{0.6\linewidth}
    \input{./images/fuente-de-tension-modelo-6.pdf_tex}
\end{center}


\subsection{Modelo $B$ de fuente de corriente}

Otra forma de simular la disminución de corriente que generaría conectar un resistor de carga a una fuente ideal, es considerar una resistencia de fuente en paralelo a una fuente ideal.

La notación con el tilde es simplemente para diferenciar los elementos de este modelo con los del anterior, que no tienen tilde.

El esquema se muestra a continuación:

\begin{multicols}{2}
    \begin{center}
        \def\svgwidth{0.9\linewidth}
        \input{./images/fuente-de-tension-modelo-3.pdf_tex}
    \end{center}
    \begin{center}
        \def\svgwidth{0.9\linewidth}
        \input{./images/fuente-de-tension-modelo-4.pdf_tex}
    \end{center}
\end{multicols}

En este caso, ambos resistores tendrían la tensión $V_F$ de la fuente ideal.
La corriente de salida $I_S'$ ahora incluiría el recorrido en paralelo.
Pero como se ve a continuación, se sigue cumpliendo que $R_L$ tiene que ser chica con respecto de $R_F$.
La corriente de salida $I_S'$ coincide con la corriente $I_S$ de la fuente del modelo anterior:
\begin{gather*}
    \left\{
    \begin{aligned}
        I_F' &= \frac{V_F'}{\frac{R_S \, R_L}{R_S + R_L}}
        \\[1ex]
        I_S' &= \frac{V_F'}{R_L}
    \end{aligned}
    \right.
    \\[1em]
    V_F' = \frac{I_F' \, R_S \, R_L}{R_S + R_L}
    \\[1em]
    I_S' = \frac{I_F' \, R_S}{R_S + R_L} = \frac{V_F}{R_S + R_L} = I_S
\end{gather*}

De esta manera, concluimos que teóricamente es posible reemplazar una fuente de corriente real con su resistencia de fuente en paralelo por una fuente de tensión real con su resistencia de fuente en serie.


\section{Divisor de tensión}

Al conectar $(\nth)$ resistores en serie a una fuente de tensión, se tiene lo que se conoce como divisor de tensión. Esta configuración permite vizualizar la caida de tensión de algún resistor mediante una fórmula.

\begin{center}
    \def\svgwidth{0.5\linewidth}
    \input{./images/divisor-de-tension.pdf_tex}
\end{center}

La corriente para cualquier elemento de un circuto en serie y perticulamente para la fuente de tensión está dada por:
\begin{equation*}
    I = \frac{V_S}{\sum_\ith^\nth R_\ith}
\end{equation*}

Entonces, la caida de tensión de cualquiera de los resistores está dada por:
\begin{equation*}
    V_{R_\ith} = I \, R_\ith = V_S \, \frac{R_\ith}{\sum_\ith^\nth R_\ith}
\end{equation*}

O bien, si $\nth=2$ se tiene:
\begin{equation*}
    \left\{
    \begin{aligned}
        V_{R_1} &= V_S \, \frac{R_1}{R_1 + R_2}
        \\[1ex]
        V_{R_2} &= V_S \, \frac{R_2}{R_1 + R_2}
    \end{aligned}
    \right.
\end{equation*}


\section{Divisor de corriente}

Al conectar $(\nth)$ resistores en paralelo a una fuente de corriente, se tiene lo que se conoce como divisor de corriente. Esta configuración permite vizualizar la corriente que pasa por algún resistor mediante una fórmula.

\begin{center}
    \def\svgwidth{0.5\linewidth}
    \input{./images/divisor-de-corriente.pdf_tex}
\end{center}

La resistencia equivalente $(R_e)$ en paralelo es:
\begin{equation*}
    \sub{R}{eq} = \frac{1}{\sum_\ith^\nth \frac{1}{R_\ith}}
\end{equation*}

La tensión para cualquier elemento de un circuito en paralelo y particularmente para la fuente de corriente está dada por:
\begin{equation*}
    V_S = I_S \, \sub{R}{eq}
\end{equation*}

La corriente que pase por algún resistor es:
\begin{equation*}
    I_\ith = \frac{V_S}{R_\ith} = \frac{I_S \, \sub{R}{eq}}{R_\ith}
\end{equation*}


\section{Superposición}

El método de superposición calcula la corriente $I_\kth$ que pasa por cada uno de los $\kth$ resistores de un circuito con $\Nth$ fuentes, ya sean de tensión o de corriente.

Primero, hay que plantear tantos circuitos como fuentes haya en el circuito original, pasivando $\Nth-1$ Fuentes en cada una de las $\Nth$ situaciones.

Cada uno de los $\Nth$ circuitos planteados va a tener una sola fuente activa y el resto pasivada. En cada una de estas situaciones, se deja activa una fuente distinta.

\begin{itemize}
    \item Para pasivar una fuente de tensión, se la reemplaza por un cable:
    \begin{equation*}
        V_S = 0 \iff R_S \to 0
    \end{equation*}

    \item Para pasivar una fuente de corriente se la desconecta, y se deja esa rama del circuito abierta:
    \begin{equation*}
        I_S = 0 \iff R_S \to \infty
    \end{equation*}
\end{itemize}

Para cada uno de los $\kth$ resistores, se calcula la corriente $I_{\kth_\nth}$ que pasa en cada uno de los $\Nth$ circuitos. Sumando las $\Nth$ corrientes de los diferentes circuitos que pasan para cierto resistor, se obtiene la corriente neta que pasa por el resistor $\kth$-ésimo:
\begin{equation*}
    I_\kth = \sum_\nth^\Nth I_{\kth_\nth} = I_{\kth_3} + I_{\kth_2} + \dots + I_{\kth_\Nth}
\end{equation*}

Donde $I_{\kth_3} + I_{\kth_2} + \dots + I_{\kth_\Nth}$ son las corrientes de cada circuito $\nth$ que pasan por un mismo resistor $\kth$.


\section{Equivalente de Thevenin}

El teorema de Thevenin sirve para diseñar un circuito compuesto por una fuente $\sub{V}{Th}$ y una resistencia $\sub{R}{Th}$ que tengan un comportamiento equivalente al de un circuito original más intrincado.

\begin{itemize}
\item Para calcular $\sub{V}{Th}$ hay que sacar $R_L$ y calcular la tensión a circuito abierto.
\item Para medir $\sub{V}{Th}$ hay que sacar $R_L$ y medir la tensión a circuito abierto.
\item Para calcular $\sub{R}{Th}$ hay que sacar $R_L$, pasivar todas las fuentes del circuito original y calcular la resistencia equivalente.
\item Para medir $\sub{R}{Th}$ hay que sacar $R_L$ y en su lugar colocar una resistencia variable $R_{POT}$ que genere una caida de tensión de $\sfrac{\sub{V}{Th}}{2}$. Al plantear el circuito de Thevenin, se tiene un divisor resistivo entre $\sub{R}{Th}$ y $R_{POT}$, a partir del cual se calcula $\sub{R}{Th}$.
\end{itemize}


\section{Potencia}

La potencia es la velocidad a la que los componentes de un circuito realizan trabajo. Para un resistor, la potencia es la cantidad de energía eléctrica por segundo que puede disipar en modo de calor:

\begin{mdframed}[style=MyFrame1]
    \begin{defn}
    \end{defn}
    \cusTi{Potencia}
    \begin{equation*}
        P = \frac{W}{\Delta t}
    \end{equation*}
\end{mdframed}

Pudiendo definir el consumo de energía electrica en \si{\kilo\watt\hour} como $W = P \Delta t$.

Por definición de potencial eléctrico $W = q \, V$, luego:
\begin{equation*}
    P = \frac{q \, V}{\Delta t}
\end{equation*}

Observar en la ecuación anterior que cargas $q$ por unidad de tiempo $\Delta t$ es intensidad de corriente, quedando la potencia definida como sigue.

\begin{mdframed}[style=MyFrame1]
    \begin{defn}
    \end{defn}
    \cusTi{Potencia}
    \begin{equation*}
        P = I \, V
    \end{equation*}
\end{mdframed}

Y aplicando la ley de Ohm, se obtienen las siguientes definiciones equivalentes entre si.

\begin{mdframed}[style=MyFrame1]
    \begin{prop}
    \end{prop}
    \begin{equation*}
        P = I \, V = \frac{V^2}{R} = I^2 \, R
    \end{equation*}
\end{mdframed}

Dado que $P = I \, V$ tanto para la resistencia interna $R_S$ de un modelo de fuente real como para la resistencia equivalente de Thevenin $\sub{R}{Th}$, la fuente del circuito equivalente va a transferirle la máxima potencia a la resistencia de carga $(R_L)$ cuando $R_L=\sub{R}{Th}$. Por ser un divisor resistivo de resistencias iguales, se tiene:
\begin{gather*}
    V_L = \frac{\sub{V}{Th}}{2}
    \\
    V_L^2 = \frac{\sub{V}{Th}^2}{4}
\end{gather*}

\begin{mdframed}[style=MyFrame1]
    \begin{prop}
    \end{prop}
    \begin{equation*}
        \sub{P}{max} = \frac{V_L^2}{\sub{R}{Th}} = \frac{\sub{V}{Th}^2}{4 \, \sub{R}{Th}}
    \end{equation*}
\end{mdframed}


\chapter{Transitorios}

\renewcommand{\iu}{\hspace{0.5mm}\mathrm{j}\mkern1mu}
\renewcommand{\voltage}{\scale{1.2}{v}}
\renewcommand{\current}{\scale{1.2}{i}}


\section{Transitorios R-C}

Los circuitos R-C son aquellos formados solamente por resistencias y capacitores.
La etapa en que un circuito R-C se estabiliza hasta tener una corriente contínua y constante se llama transitoria.

\begin{center}
    \def\svgwidth{0.5\linewidth}
    \input{./images/transitorios-capacitor-circuito.pdf_tex}
\end{center}

Inicialmente, el capacitor comienza a cargarce.
Mientras no haya tanto amontonamiento de cargas, una placa del capacitor puede recibir electrones y la otra puede sederlos.
Por esto, el circuito se comporta como si hubiese corriente contínua.
Al principio, la corriente es máxima y luego va decayendo.
Mientras que la tensión en el capacitor va a ir creciendo conforme pase el tiempo, hasta igualar la tensión en fuente.
Este comportamiento es representado por la siguiente ecuación diferencial.
\begin{equation*}
    \current_C(t) = C \, \frac{\dif}{\dif t} \voltage_C(t)
\end{equation*}

Cuya solución es para la corriente
\begin{equation*}
    \current_C(t) = \frac{V_S}{R} \, e^{-\tfrac{t}{\tau}}
\end{equation*}

\begin{center}
    \def\svgwidth{0.8\linewidth}
    \input{./images/transitorios-capacitor-graph-1.pdf_tex}
\end{center}

Por ley de Kirchhoff la tensión es
\begin{equation*}
    \voltage_C(t) = V_S - \voltage_R(t) = V_S - R \, \current(t)
\end{equation*}

Y ya que $\current(t) = \current_C(t)$ se obtiene
\begin{equation*}
    \voltage_C(t) = V_S - V_S \, e^{-\tfrac{t}{\tau}}
\end{equation*}

\begin{center}
    \def\svgwidth{0.8\linewidth}
    \input{./images/transitorios-capacitor-graph-2.pdf_tex}
\end{center}

De manera general, si el capacitor tiene una tensión inicial $(V_0)$, la expresión que modela la tensión es:

\begin{equation*}
    \voltage_C(t) = V_1 - \bb{V_1 - V_0} e^{-\tfrac{t}{\tau}}
\end{equation*}

Donde la tensión final es $V_1 = V_S$ para la etapa de carga y $V_1 = 0$ para la etapa de descarga.

Si la resistencia $(R)$ o la capacidad $(C)$ son grandes, el tiempo $(\tau)$ de cargado del capacitor tiene que ser mayor. Cuando $t = 5 \, \tau$ el capacitor tiene el $99\%$ de la tensión de la fuente, por lo que se considera cargado. Cada $\tau$ segundos la corriente disminuye aproximadamente el $33\%$.

\begin{mdframed}[style=MyFrame1]
    \begin{defn}
    \end{defn}
    \begin{equation*}
        \tau = R \, C
    \end{equation*}
\end{mdframed}

En el circuito dado, el capacitor está obligado a estar o bien cargandose conectado a la fuente o bien descargandose.
La llave del interruptor cambia con una frecuencia constante.
Si $\tau$ es lo suficientemente chico, dará tiempo al capacitor de cargarse y descargarse mucho antes que la llave se vuelva a cambiar, y el gráfico de la tensión tendrá forma cuadrada, de lo contrario triangular.

\begin{center}
    \def\svgwidth{\linewidth}
    \input{./images/transitorios-capacitor.pdf_tex}
\end{center}


\section{Transitorios R-L}

Al contrario que en un circuito R-C, en la bobina de un circuito R-L, al principio la corriente es mínima y la tensión máxima.
\begin{equation*}
    \voltage_L(t) = L \, \frac{\dif}{\dif t} \current_L(t)
\end{equation*}

La corriente en una bobina se comporta como la tensión en un capacitor:
\begin{equation*}
    \current_L(t) = I_1 - \bb{I_1 - I_0} e^{-\tfrac{t}{\tau}}
\end{equation*}

Y la tensión como la corriente:
\begin{equation*}
    \voltage_L(t) = V_S \, e^{-\tfrac{t}{\tau}}
\end{equation*}

Quedando la constante de tiempo definida como sigue.

\begin{mdframed}[style=MyFrame1]
    \begin{defn}
    \end{defn}
    \begin{equation*}
        \tau = \frac{L}{R}
    \end{equation*}
\end{mdframed}


\chapter{Circuitos AC}


\section{CIVIL}

En un resistor no hay defasaje para los fasores de su tensión y corriente, pero para un capacitor y un inductor se tiene, respectivamente:
\begin{gather*}
    C: \left\{
    \begin{aligned}
        \phi_{\voltage_C} &= \phi_{\current_C} - \ang{90}
        \\
        \phi_{\current_C} &= \phi_{\voltage_C} + \ang{90}
    \end{aligned}
    \right.
    \\[1em]
    L: \left\{
    \begin{aligned}
        \phi_{\voltage_L} &= \phi_{\current_L} + \ang{90}
        \\
        \phi_{\current_L} &= \phi_{\voltage_L} - \ang{90}
    \end{aligned}
    \right.
\end{gather*}

A continuación vemos esta variación de la tensión y la corriente en el tiempo, representada con fasores complejos.

\begin{center}
    \def\svgwidth{0.6\linewidth}
    \input{./images/ac-fase.pdf_tex}
\end{center}

Una interpretación intuitiva del comportamiento de un capacitor se representa con un tanque de agua con capacidad para almacenar volumen de agua, que tiene una válvula que puede suministrar un flujo de agua.

\begin{center}
    \def\svgwidth{0.9\linewidth}
    \input{./images/ac-capacitor-tanque.pdf_tex}
\end{center}

Graficando las etapas de carga y descarga, si se toma un intervalo de tiempo infinitesimal, las gráficas discretas aproximan a funciones senoidales:

\begin{center}
    \def\svgwidth{0.8\linewidth}
    \input{./images/ac-capacitor-fase.pdf_tex}
\end{center}

La analogía infiere que, si la corriente es el flujo de agua que entra o sale, y la tensión es el volumen de agua que se acumula, entonces la corriente es la tasa de cambio de la tensión:
\begin{equation*}
    \current_C(t) = C \, \frac{\dif}{\dif t} \voltage_C(t)
\end{equation*}

Mientras que para un inductor se tiene:
\begin{equation*}
    \voltage_L(t) = L \, \frac{\dif}{\dif t} \current_L(t)
\end{equation*}


\section{Impedancia}

Se llaman reactancia capacitiva ($X_C$) y reactancia inductiva ($X_L$) a las magnitudes que representan la oposición al paso de corriente impuesta por los capacitores e inductores respectivamente.

\begin{mdframed}[style=MyFrame1]
    \begin{defn}
    \end{defn}
    \cusTi{Reactancia capacitiva}
    \begin{equation*}
        X_C = \frac{1}{\omega \, C} = \frac{1}{2 \, \pi \, f \, C}
    \end{equation*}
\end{mdframed}

\begin{mdframed}[style=MyFrame1]
    \begin{defn}
    \end{defn}
    \cusTi{Reactancia inductiva}
    \begin{equation*}
        X_L = \omega \, L = 2 \, \pi \, f \, L
    \end{equation*}
\end{mdframed}

La impedancia ($\impedance$) engloba la resistencia ($R$), la reactancia capacitiva ($X_C$), y la reactancia inductiva ($X_L$).

\begin{mdframed}[style=MyFrame1]
    \begin{defn}
    \end{defn}
    \cusTi{Impedancia}
    \begin{equation*}
        \impedance = R + \iu \bb{X_L - X_C}
    \end{equation*}
\end{mdframed}

De manera que la Ley de Ohm se sigue cumpliendo, teniendo en cuenta que la impedancia es una división entre fasores complejos.
\begin{equation*}
    \impedance = \frac{\voltage}{\current}
    = \frac{V \, e^{\iu \bb{\omega \, t + \phi_\voltage}}}{I \, e^{\iu \bb{\omega \, t + \phi_\current}}}
    = \frac{V}{I} \, e^{\iu \bb{\phi_\voltage - \phi_\current}}
    = \frac{V}{I} \, e^{\iu \, \phi_\impedance}
    = Z \, e^{\iu \, \phi_\impedance}
\end{equation*}

\begin{mdframed}[style=MyFrame1]
    \begin{prop}
    \end{prop}
    Para un componente púramente resistivo se tiene que $\phi_{\voltage_R} - \phi_{\current_R} = \phi_{\impedance_R} = 0$ tal que
    \begin{equation*}
        \impedance_R = \frac{V_R}{I_R} \, e^0 = R
    \end{equation*}
\end{mdframed}

\begin{mdframed}[style=MyFrame1]
    \begin{prop}
    \end{prop}
    Para un componente púramente capactivo se tiene que $\phi_{\voltage_C} - \phi_{\current_C} = \phi_{\impedance_C} = - \ang{90}$ tal que
    \begin{equation*}
        \impedance_C = \frac{V_C}{I_C} \, e^{- \iu \, \frac{\pi}{2}} = - \iu \, X_C
    \end{equation*}
\end{mdframed}

\begin{mdframed}[style=MyFrame1]
    \begin{prop}
    \end{prop}
    Para un componente púramente inductivo se tiene que $\phi_{\voltage_L} - \phi_{\current_L} = \phi_{\impedance_L} = \ang{90}$ tal que
    \begin{equation*}
        \impedance_L = \frac{V_L}{I_L} \, e^{\iu \, \frac{\pi}{2}} = \iu \, X_L
    \end{equation*}
\end{mdframed}


\section{Circuitos R-C en serie}

\begin{equation*}
    \sub{\impedance}{eq} = \impedance_R + \impedance_C
\end{equation*}

Donde:
\begin{align*}
    \sub{Z}{eq} &= \sqrt{R^2 + X_C^2}
    \\
    \phi_{\sub{\impedance}{eq}} &= \arctan{\bb{\dfrac{-X_C}{R}}}
\end{align*}

\begin{align*}
    \current &= \frac{\voltage_s}{\sub{\impedance}{eq}}
    \\
    &= \frac{V_s}{\sub{Z}{eq}} \, e^{\iu \bb{\ang{0} - \phi_{\sub{\impedance}{eq}}}}
\end{align*}

Donde:
\begin{align*}
    I &= \frac{V_s}{\sub{Z}{eq}}
    \\
    \phi_i &= - \phi_{\sub{Z}{eq}}
\end{align*}

\begin{align*}
    \voltage_C &= \current \, \impedance_C
    \\
    &= \frac{V_s \, X_C}{\sub{Z}{eq}} \, e^{\iu \phi_{\voltage_C}}
\end{align*}

Donde:
\begin{align*}
    V_C &= \frac{V_s \, X_C}{\sqrt{R^2 + X_C^2}}
    \\
    \phi_{\voltage_C} &= \phi_\current + \phi_{\impedance_C} = - \arctan{\bb{\dfrac{-X_C}{R} }} - \ang{90}
\end{align*}


\section{Potencia}

La potencia en AC es similar a la definida para DC. La diferencia es que en alterna, la tensión y la corriente son funciones del tiempo. De manera que la potencia también lo es.

\begin{mdframed}[style=MyFrame1]
    \begin{defn}
    \end{defn}
    \cusTi{Potencia en AC}
    \begin{equation*}
        \fx{p}{t} = \fx{\current}{t} \, \fx{\voltage}{t}
    \end{equation*}
\end{mdframed}

La tensión y la corriente son ondas senoidales de igual frecuencia, pero que pueden estár defasadas entre sí. Sea $\varphi$ el defasaje entre $\fx{\current}{t}$ y $\fx{\voltage}{t}$ podemos expresar la potencia como
\begin{align*}
    \fx{p}{t} &= I \, \sin (\omega \, t) \, V \, \sin (\omega \, t + \varphi)
    \\[1ex]
    &= V \, I \, \cos (\varphi) \sqb{1 - \cos(2 \, \omega \, t)} + V \, I \, \sin (\varphi) \sin (2 \, \omega \, t)
\end{align*}


\section{Filtros}

Al aumentar la frecuencia de la corriente en un circuito R-C, el capacitor tiene menos tiempo de cargado por ciclo. Por más que la intensidad de corriente máxima sea la misma, la tensión máxima va a ser menor.

Un filtro es un divisor de tensión que tiene una rama \emph{en serie} y otra \emph{en paralelo} o \emph{en derivación}. En cada rama puede tener uno o más componentes, que pueden ser resistores, capacitores o inductores dependiendo la aplicación:

\begin{center}
    \def\svgwidth{0.6\linewidth}
    \input{./images/filter.pdf_tex}
\end{center}

La frecuencia de corte o frecuencia de cruce $f_c$ es el aquella para la que ambas ramas tienen igual impedancia.

El siguiente gráfico muestra las curvas de impedancia de un resistor $(R)$, un capacitor $(C)$ y una bobina $(L)$ para diferentes frecuencias:

\begin{center}
    \def\svgwidth{0.7\linewidth}
    \input{./images/filter-fc.pdf_tex}
\end{center}


\subsection{Filtros R-C}

El siguiente es un filtro R-C pasa altos. Invirtiendo la disposición de los componentes, se tiene un filtro R-C pasa bajos.

\begin{center}
    \def\svgwidth{0.6\linewidth}
    \input{./images/filter-rc.pdf_tex}
\end{center}

La frecuencia central es aquella que verifica $R = X_C$ y está dada por:
\begin{equation*}
    f_c = \frac{1}{2 \, \pi \, C \, R}
\end{equation*}


\subsection{Función de transferencia}

La función de transferencia es
\begin{equation*}
    H = \frac{\sub{\voltage}{out}}{\sub{\voltage}{in}}
    = \frac{\iu \, R}{\iu \, \sub{Z}{eq}}
    = \frac{R}{R - \iu \, X_C}
\end{equation*}

o bien
\begin{equation*}
    H = \frac{1}{1 - \iu \, \tfrac{1}{\omega \, R \, C}}
\end{equation*}

La frecuencia de resonancia verifica $Z_R = Z_C$ o bien $R = \frac{1}{\omega_0 \, C}$, quedando la frecuencia de cruce
\begin{equation*}
    \omega_0 = \frac{1}{R \, C}
\end{equation*}

Y la función de transferencia
\begin{equation*}
    H = \frac{1}{1 - \iu \, \tfrac{\omega_0}{\omega}} = \frac{1}{\sqrt{1 +  \bb{\tfrac{\omega_0}{\omega}}^2}} \, e^{\iu \artan \bb{\tfrac{\omega_0}{\omega}}}
\end{equation*}

De manera que si $\omega = \omega_0$ el circuito está en resonancia y
\begin{equation*}
    \norm{H} = \frac{1}{\sqrt{2}} = \frac{\sub{V}{out}}{\sub{V}{in}} \approx 0.707
\end{equation*}

Que en decibeles es
\begin{align*}
    L_H &= 20 \, \log \bb{\frac{\sub{V}{out}}{\sub{V}{in}}}
    \\[1ex]
    &= -20 \, \log \sqrt{1 + \bb{\frac{\omega_0}{\omega}}^2}
    \\[1ex]
    &= -10 \, \log \sqb{1 + \bb{\frac{\omega_0}{\omega}}^2}
\end{align*}

Ahora bien, es posible hacer una aproximación que tenga una caída de $6 \, \si{\deci \bel}$ por octava, sabiendo que la atenuación real en la frecuencia de resonancia va a ser de $3 \, \si{\deci \bel}$.

Esto es, considerando $f << f_0$ se tiene que $1 << \bb{\tfrac{\omega_0}{\omega}}^2$ y se obtendría para la función de transferencia
\begin{equation*}
    L_H \approx 20 \, \log \bb{\frac{\omega}{\omega_0}}
\end{equation*}

Pudiendo comparar ambas respuestas en frecuencia en el siguiente gráfico

\begin{center}
    \def\svgwidth{\linewidth}
    \input{./images/filtros-func-transfer.pdf_tex}
\end{center}


\subsection{Filtros R-L}

El siguiente es un filtro R-L pasa bajos. Invirtiendo la disposición de los componentes, se tiene un filtro R-L pasa altos.
\begin{equation*}
    R = X_L \Rightarrow f_c = \frac{R}{2 \, \pi \, L}
\end{equation*}

\begin{center}
    \def\svgwidth{0.6\linewidth}
    \input{./images/filter-rl.pdf_tex}
\end{center}


\subsection{Filtros L-C}

El siguiente es un filtro L-C pasa bajos. Invirtiendo la disposición de los componentes, se tiene un filtro L-C pasa altos.
\begin{equation*}
    X_C = X_L \Rightarrow f_c = \dfrac{1}{2 \, \pi \, \sqrt{L \, C}}
\end{equation*}

\begin{center}
    \def\svgwidth{0.6\linewidth}
    \input{./images/filter-lc.pdf_tex}
\end{center}


\subsection{Filtro pasa banda}

El siguiente es un filtro pasa banda. Invirtiendo la disposición de los componentes, se tiene un filtro rechaza banda.
\begin{align*}
    \norm{\impedance_R} &= \norm{\impedance_C + \impedance_L}
    \\
    R &= \norm{\iu \, \omega \, L - \iu \, \frac{1}{\omega \, C}}
    \\
    &= \omega \, L - \frac{1}{\omega \, C}
    \\
    &= 2 \, \pi \, f_c \, L - \frac{1}{2 \, \pi \, f_c \, C}
\end{align*}

\begin{center}
    \def\svgwidth{0.6\linewidth}
    \input{./images/filter-rlc.pdf_tex}
\end{center}

La pendiente es proporcional a la resistencia:
\begin{equation*}
    Q = \frac{f_c}{\Delta f} \propto R
\end{equation*}


\section{Transformadores}

El transformador sirve para aumentar o disminuir la tensión y la corriente de una fuente AC.

Está formado por dos bobinas, a lo largo de las cuales hay un material que permite el flujo magnético. Cuando circula corriente por una de las bobinas, se genera un campo magnético que es detectado por la segunda bobina, y genera un campo independiente del anterior. Este va a generar una corriente análoga a la de entrada, pero cuya tensión $(V)$ e intensidad $(I)$ va a depender de la cantidad de vueltas que ambas bobinas tengan.

\begin{center}
    \def\svgwidth{\linewidth}
    \input{./images/transformador.pdf_tex}
\end{center}

La función del transformador es, como se mencionó anteriormente, modificar la tensión, lo cual va a afectar la corriente. Por la ley del transformador ideal, lo que permanece constante es la potencia:
\begin{equation*}
    V_0 \, I_0 = V_1 \, I_1
\end{equation*}

\begin{mdframed}[style=MyFrame2]
    \begin{example}
    \end{example}
    \cusTi{Lámpara EUR-USA}
    \begin{formatI}
        Con frecuencia se suelen usar transformadores para conectar un dispositivo que esté diseñado para ser usado con la tensión que se usa en el país de procedencia del dispositivo. En Argentina y Europa en general, se utiliza $\SI{220}{\volt}$ y en Estados Unidos $\SI{110}{\volt}$. Verificar que una lámpara de $\SI{55}{\watt}$ de Argentina no va a ser igual que una lámpara de $\SI{55}{\watt}$ de Estados unidos, pero ambas van alumbrar con la misma potencia.
    \end{formatI}
    
    Planteando $P = V \, I$ para cada país, se tiene:
    \begin{align*}
        \SI{55}{\watt} = \SI{220}{\volt} \times \sub{I}{Arg} & \Rightarrow \sub{I}{Arg} = \SI{0.25}{\ampere}
        \\
        \SI{55}{\watt} = \SI{110}{\volt} \, \sub{I}{USA} & \Rightarrow \sub{I}{USA} = \SI{0.5}{\ampere}
    \end{align*}
    
    En conclusión, la lámpara de $55 watts$ argentina va a proporcionar una resistencia de $880\Omega$ mientras que la de Estados Unidos $220\Omega$.
    \begin{equation*}
        \sub{R}{Arg} = \SI{880}{\ohm} \neq \sub{R}{USA} = \SI{220}{\ohm}
    \end{equation*}
    
    Es decir, que si se conecta una lámpara argentina en la red de Estados Unidos va a alumbrar menos que si se conecta en Argentina, y si se conecta una lámpara de Estados Unidos en Argentina sin transformador se quema.
\end{mdframed}


\section{Fuente AC/DC}

Supongamos que se tiene un tanque al que, con una frecuencia constante, por momentos se le agrega agua y por momentos se le quita. A partir de esa situación, se quiere instalar una canilla por la que salga una corriente de agua constante. En principio se necesita que el llenado de agua solo sea entrante, evitando que se descargue. Y además, se necesita que haya cierta acumulación de agua en el tanque para que, si abro la canilla más de lo que la entrada está aportando, no baje la presión de manera sustancial. Para convertir una fuente de AC en DC, el procedimiento es similar a la analogía anterior.

\begin{itemize}
    \item \concept{Fuente AC:} suministra de elergía eléctrica el dispositivo.
    \item \concept{Transformador:} adapta la tensión que entrega la red. En el caso de argentina la entrada del transformador sería de $\SI{220}{\volt}$ y la salida depende del equipo en cuestión.
    \item \concept{Etapa de rectificación:} hace que la corriente tenga solo un sentido. Suelen usarse diodos.
    \item \concept{Etapa de filtrado:} obtiene una acumulación de cargas para que las cargas que salgan no superen a las que entren cortando el flujo de corriente. Suelen usarse capacitores.
    \item \concept{Resistencia:} carga equivalente al dispositivo que estamos conectando a la fuente.
\end{itemize}

\begin{center}
    \def\svgwidth{\linewidth}
    \input{./images/fuente-simple.pdf_tex}
\end{center}

\concept{Puente de diodos}

Para hacer la etapa de rectificación más eficiente, se necesitaría aprovechar cuando la corriente cambia de sentido. Es decir, en vez de solo dejar pasar la corriente cuando es positiva, de alguna forma ``cambiar los extremos de los cables'' cuando la fuente alterna cambie de polaridad de manera que la polaridad resultante siempre sea la misma.

\begin{center}
    \def\svgwidth{0.6\linewidth}
    \input{./images/puente-de-diodos-1.pdf_tex}
\end{center}

\begin{center}
    \def\svgwidth{0.6\linewidth}
    \input{./images/puente-de-diodos-2.pdf_tex}
\end{center}

Claro está que realizar este cambio manualmente es imposible. Pero la siguiente configuración, llamada Puente de Diodos cumple la función de hacer la corriente siempre positiva.

\begin{center}
    \def\svgwidth{0.6\linewidth}
    \input{./images/puente-de-diodos-3.pdf_tex}
\end{center}


\end{document}